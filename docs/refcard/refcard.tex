%-------------------------------------------------------------------------------

% This file is part of Code_Saturne, a general-purpose CFD tool.
%
% Copyright (C) 1998-2020 EDF S.A.
%
% This program is free software; you can redistribute it and/or modify it under
% the terms of the GNU General Public License as published by the Free Software
% Foundation; either version 2 of the License, or (at your option) any later
% version.
%
% This program is distributed in the hope that it will be useful, but WITHOUT
% ANY WARRANTY; without even the implied warranty of MERCHANTABILITY or FITNESS
% FOR A PARTICULAR PURPOSE.  See the GNU General Public License for more
% details.
%
% You should have received a copy of the GNU General Public License along with
% this program; if not, write to the Free Software Foundation, Inc., 51 Franklin
% Street, Fifth Floor, Boston, MA 02110-1301, USA.

%-------------------------------------------------------------------------------

\documentclass[a4paper,11pt]{article}

\usepackage{hyperref}
\usepackage[landscape]{geometry}
\usepackage{fullpage}
\usepackage{multicol}
\usepackage{graphicx}
\usepackage{xspace}
\input{csvers}

\pagestyle{empty}

\setlength\columnseprule{0.4pt}
\addtolength\columnsep{2pt}

\newcommand{\refword}[1]{\texttt{$\bullet$ \bf{#1}}}

\begin{document}

\begin{multicols*}{3}

\begin{center}
  \includegraphics[width=4cm]{cs_logo_flux}
  {\Large {\bf \CS~\verscs\\Quick reference card}}
\end{center}

% User scripts
% ------------

\section*{User scripts}

All \CS commands are available under a single script: \texttt{code\_saturne}.
Here below are the most useful commands for a \CS user from the study
creation to the post-processing. Complete information for each
\texttt{command} can be obtained by typing:\\
\texttt{code\_saturne <command> --help}.\\

\refword{info}\\
Get information on \CS. Open the documentation (user, theory).\\
\textit{e.g.} \texttt{code\_saturne info --guide \emph{user}}\\

\refword{config}\\
Get information on the configuration and installation of \CS.\\
\textit{e.g.} \texttt{code\_saturne config}\\

\refword{create}\\
Create a \CS template study or case.\\
\textit{e.g.} \texttt{code\_saturne create --study \emph{study1}}\\

\refword{gui}\\
Launch \CS graphical user interface.\\
\textit{e.g.} \texttt{code\_saturne gui --file \emph{xmlfile}}\\

\refword{trackcvg}\\
Launch \CS track convergence tool.\\
\textit{e.g.} \texttt{code\_saturne trackcvg RESU/run\_name}\\

\refword{compile}\\
Create a specific solver executable when some user subroutines are
present.\\
\textit{e.g.} \texttt{code\_saturne compile --test}\\

\refword{studymanager}\\
Launch studymanager.\\
\textit{e.g.} \texttt{code\_saturne studymanager --file \emph{xmlfile} --run}\\


% User subroutines
% ----------------

\section*{Main user subroutines}

Here below are the most useful user subroutines to run a standard
simulation. Some of them are useless if the graphical user interface
is used.\\

\refword{cs\_user\_parameters.c}\\
Initialization of the main keywords.\\

\refword{cs\_user\_boundary\_conditions.f90}\\
Management of the boundary conditions.\\

\refword{cs\_user\_physical\_properties.c}\\
Management of the variable physical properties.\\

\refword{cs\_user\_initialization.c}\\
Non-standard initialization of the variables.\\

\refword{cs\_user\_extra\_operations.c}\\
User project files.\\

\refword{cs\_user\_head\_losses.c}\\
Management of the head loss.\\

\refword{cs\_user\_source\_terms.c}\\
User source terms related subroutines.\\

\refword{cs\_user\_postprocess.c}\\
Post-processing related subroutines.\\

% Main variables
% --------------

%\section*{Main variables}

% Practical information
% ---------------------

\section*{Practical information}

\url{https://www.code-saturne.org}\\

Related software:\\
\url{https://www.salome-platform.org}

\begin{center}
  \includegraphics[width=2cm]{logoedf}
\end{center}

\end{multicols*}

\end{document}
